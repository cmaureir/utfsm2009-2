%Organización del grupo de trabajo, miembros, roles asumidos y trabajo realizado por cado uno de ellos

\subsection{Sobre la organización}

La organización del presente equipo de trabajo se basa en una estructura piramidal,
teniendo un \emph{Jefe de Proyecto} a cargo de organizar, coordinar y verificar
las actividades realizadas durante todo el semestre.

Los medios de comunicación oficiales para el desarrollo del proyecto,
fueron un \emph{alias de correo}~\footnote{tsc@csrg.inf.utfsm.cl} para realizar los comunicados oficiales,
tareas pendientes, división de tareas para la realización de informes, etc,
medio que cumplió un rol fundamental para que cada miembro se enterara
de cada uno de los hitos a lo largo del desarrollo del presente proyecto.

Otro medio de comunicación fue \emph{Skype}~\footnote{http://www.skype.com} que ayudó fielmente a la discusión
de ciertas ideas a lo largo del proyecto, como por ejemplo, el cómo implementar
ciertos servicios, configuraciones del servidor que se utilizó para el trabajo,
analizar tareas realizadas por el resto del equipo, etc.

Finalmente el papel que jugó nuestro \emph{TRAC}~\footnote{https://trac.inf.utfsm.cl/proyectos/tsc\_02/},
fue la espina dorsal de nuestro proyecto, pues es el medio oficial en el cual plasmábamos
nuestro trabajo con tutoriales, documentos relacionados con el \emph{Control de Acceso Lógico},
instructivos, etc, todo tipo de documentos e información relevante para una correcta implementación
de nuestra tarea asignada para éste semestre.


\subsection{Sobre los miembros}

Cada miembro del equipo de trabajo tenía como responsabilidad cierto rol,
en función directa a la realización de las actividades planificadas en el
semestre.

Si bien es cierto, las personas encargadas de ciertas actividades, debían
velar por que todo se desarrollara de la mejor forma posible, también es
importante recalcar que todos estuvimos dentro de las tomas de decisiones
y puesta en práctica de las otras actividades, para poder saber completamente
el estado general del grupo y las actividades que todo el equipo realizaba.

A continuación una tabla con los cargos de cada uno:

\begin{center}
	\begin{tabular}{|c|c|}
		\hline
		Esteban Bombal    & Jefe de Proyecto \\\hline
		Rodrigo Fernández & Área Documentación \\\hline
		Ignacio Villacura & Área Técnica \\\hline
		Gabriel Zamora    & Área Documentación \\\hline
		Cristián Maureira & Área Técnica \\\hline
	\end{tabular}
\end{center}

\subsection{Sobre el trabajo}

Durante el semestre nuestro equipo de trabajo poseía una \emph{Carta Gantt} la cual fue la guía
para la realización de todas las actividades necesarias.

Algunas de las principales tareas a lo largo del desarrollo de nuestro proyecto fueron:

\begin{description}
	\item[Investigación sobre el tema:]
		\begin{itemize}
			\item \texttt{Encargados:} Todos
			\item \texttt{Descripción:}
				Durante el proceso de Investigación el grupo se centro en comprender el funcionamiento
				de autenticación mediante \emph{JAAS}, los beneficios que dicha aplicación nos ofrece y la
				escalabilidad que otorga, entre ellos el \emph{Single Sign-On}, sin dejar de lado otras temáticas
				como lo fueron \emph{LDAP}, \emph{JBoss}, etc.
		\end{itemize}
	\item[Implementación de nuestra máquina de trabajo:]
		\begin{itemize}
			\item \texttt{Encargados:} Todos
			\item \texttt{Descripción:}
				Se realizó la instalación de una máquina de trabajo desde cero,
				en el Laboratorio de Computación (LabComp)~\footnote{http://labcomp.inf.utfsm.cl}
				con el sistema operativo \emph{Fedora 11} para poder realizar todas nuestro
				trabajo.
		\end{itemize}
	\item[Implementación Servidor JBoss:]
		\begin{itemize}
			\item \texttt{Encargados:} Cristián Maureira y Gabriel Zamora
			\item \texttt{Descripción:}
				Fue necesario poder comprender el funcionamiento de las distintas aplicaciones
				que los demás proyectos se encontraban realizando, como lo fue el caso de un servidor JBoss.
				Por lo cual tuvimos que implementar nuestro propio servidor JBoss para poder hacer las pruebas
				correspondientes, que ayudaran a realizar nuestro trabajo.
		\end{itemize}
	\item[Integración JAAS y JBoss:]
		\begin{itemize}
			\item \texttt{Encargados:} Ignacio Villacura y Esteban Bombal
			\item \texttt{Descripción:}
				Se realizaron todas las actividades necesarias para poder tener en funcionamiento
				JAAS dentro de nuestro servidor JBoss.
		\end{itemize}
	\item[Integración LDAP en JBoss mediante JAAS:]
		\begin{itemize}
			\item \texttt{Encargados:} Rodrigo Fernández
			\item \texttt{Descripción:}
				Se instaló LDAP en nuestra máquina de prueba, con algunos
				usuarios de prueba, para poder probar el funcionamiento
				de JAAS en nuestro servidor JBoss.
		\end{itemize}
	\item[Implementación de Single Sign-On en servidor JBoss:]
		\begin{itemize}
			\item \texttt{Encargados:} Gabriel Zamora, Ignacio Villacura y Cristián Maureira
			\item \texttt{Descripción:}
				Se realizaron las configuraciones necesarias para poder utilizar
				Single Sign-On en nuestro servidor JBoss, las cuales contaron
				con la modificación de algunos archivos de configuración de JBoss
				para que reconociera dicha función.
		\end{itemize}
	\item[Integración con Tomcat:]
		\begin{itemize}
			\item \texttt{Encargados:} Esteban Bombal y Rodrigo Fernández
			\item \texttt{Descripción:}
				Se realizaron las configuraciones necesarias para poder autentificar en un servidor
				Tomcat mediante JAAS.
		\end{itemize}
\end{description}
