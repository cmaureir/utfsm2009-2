%Sugerencias y recomendaciones de mejora para el futuro.

Si bien es cierto, el objetivo principal del ramo en general fue correcto;
pequeños proyectos desarrollados por distintos grupos y al final poder
realizar una integración de todos los proyectos, siendo esta una manera
muy profesional de trabajar, contempla una correcta sincronía y óptima comunicación
entre todos los grupos en el ramo, el cual es un punto crítico, pues como se nombro
anteriormente es difícil que distintos proyectos estén en una completa comunicación
debido a la actividad que coincidió en el semestre, la Feria de Software.

Sería bueno reformular el objetivo final del ramo de una integración global,
a integración por sectores, es decir, integración de 2 o 3 grupos como máximo,
con lo cuál tendremos subconjuntos de grupos que pueden tener una orientación
distintas unos de otro, sin tener tanta dependencia entre ellos.

Quizás otro punto importante es poder ahondar en otros temas y lenguajes de programación,
si bien es cierto \emph{JAVA} es un lenguaje muy utilizado en las empresas de hoy,
hay otras herramientas que pueden ser de utilidad de la misma forma con otros lenguajes,
para tener al final sub-proyectos que abarquen distintas áreas de desarrollo,
ya sean \emph{CMS} con autenticaciones (Joomla, Drupal), Aplicaciones Web con \emph{Django} (Python),
Herramientas \emph{.NET} ya sea \emph{C\#, ASP}, etc.

Finalmente la elección de temas debería realizarse de otra forma,
el poseer un sistema de \emph{Enviar un mail y por orden de llegada se asignan} es poco
óptimo, por lo cual en vez de designar los temas de esa manera, se podrían dar
los temas como ideas, y que cada grupo deba debatir un proyectos relacionado a
uno de ellos, explicando el \emph{¿Por qué?} de la elección del tema,
exponiendo sus razones de querer trabajar con dicho tema,
y después de recibir las propuestas de cada uno, pasar a aprobar el proyecto o
rectificarlo en caso de ser muy simple o muy complejo. El asignar temas fijos
y que queden grupos con temáticas muy especificas que no son del interés del
grupo es desmotivante y hace que no se realicen muy buenos proyectos.
