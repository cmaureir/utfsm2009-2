%Planificación y ejecución del proyecto. Evaluación del cumplimiento de objetivos y plazos.
La planificación realizada tuvo 3 puntos importantes,
investigación, implementación e interacción,
y 2 que se refieren a los entregables y a la mantención
de la bitácora y carta Gantt.

A continuación detallaremos como fue la ejecución del proyecto:

\begin{itemize}

	\item Investigación(Sept. 8 - Sept. 18): En esta parte no tuvo contra-tiempos
		y se desarrollo con normalidad. Se puso bastante énfasis en documentar
		en nuestro TRAC, todo lo necesario acerca de JAAS.

	\item Implementación(Sept. 18 - Nov. 12): Para esta parte del proyecto se disponía
		de poco tiempo ya que todos los integrantes del grupo participaban de la
		\emph{Feria de Software}, pero aun así se lograron todas las metas propuestas
		en este punto. También se consumió mas tiempo ya que tuvimos que implementar
		nuestros propios servidores, ya sea JBOSS o Tomcat.

	\item Interacción(Nov. 12 - Nov. 16): Este punto es el mas problemático, si bien en
		nuestros servidores funcionan todas las implementaciones, por falta de tiempo
		de los integrantes debido a la carga académica, fue imposible hacer la integración
		con el grupo de \emph{Servidores Web} y \emph{Servidor JBOSS},  también era nuestra
		intención utilizar los servicios del grupo \emph{LDAP}.

	\item Entregables(Sept. 21 - Nov. 20): Estos ocurrieron sin ningún problema. Todos se
		hicieron en los tiempo adecuados al igual que las presentaciones, menos una en la
		cual algunos integrantes estaban atendiendo a las \emph{Jornadas Chilenas de la
		Computación} y se realizó después en común acuerdo con el profesor y con los integrantes.

	\item Mantención(Sept. 8 - Nov.16): La bitácora se encuentra en el TRAC donde están todas las
		cosas que se hicieron, ya sea Investigación e Implementación. Por la parte de la Carta
		Gantt, se fue actualizando a medida que tuviésemos algún atraso, ésto no ocurrió ya que
		todo se alcanzo a hacer en los tiempo correctos, a excepción de la integración con los otros equipos. 
\end{itemize}
