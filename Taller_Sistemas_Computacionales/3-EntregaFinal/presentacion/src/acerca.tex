% Algunas definiciones para que se entienda el coontenido

\frame{
\frametitle{Acerca}
\framesubtitle{Objetivos}
\begin{itemize}
	\item Instalación de API JAAS
	\item Integración de LDAP en JBoss usando JAAS
	\item Integración de JAAS en Tomcat
	\item Implementación de Single Sign-On
\end{itemize}
}

\frame{
\frametitle{Acerca}
\framesubtitle{Conceptos}
\textbf{JAAS}
\begin{itemize}
	\item Java Authentication and Authorization Service
	\item Interfaz de Programación de Aplicaciones
	\item Permite a las aplicaciones Java acceder a servicios de control de autenticación y acceso.
\end{itemize}
}

\frame{
\frametitle{Acerca}
\framesubtitle{Conceptos}
\textbf{JBoss}
\begin{itemize}
	\item Es un servidor de aplicaciones J2EE de código abierto implementado en Java puro.
	\item Como es basado en Java, puede ser utilizado en cualquier SO que lo soporte.
	\item Implementa todo el paquete de servicios de J2EE.
\end{itemize}
}

\frame{
\frametitle{Acerca}
\framesubtitle{Conceptos}
\textbf{SSO}
\begin{itemize}
	\item Single Sign-On o Reduced Sign-on Systems
	\item Procedimiento de autenticación
	\item Habilita al usuario para acceder a varios sistemas con una sola instancia de identificación.
	\item Hay cinco tipos principales de SSO:
	\begin{itemize}
		\item Enterprise Single Sign-On (E-SSO)
		\item Web Single Sign-On (Web-SSO)
		\item Kerberos
		\item Identidad Federada
		\item OpenID
	\end{itemize}
\end{itemize}
}


\frame{
\frametitle{Acerca}
\framesubtitle{Conceptos}
\textbf{Servlet}
\begin{itemize}
	\item Objetos que corren dentro de un contenedor de servlets (ej: Tomcat) y extienden su funcionalidad.
	\item También dentro de un servidor de aplicaciones.
	\item ...además tendrá contenedor para objetos más avanzados (ej: EJB).
\end{itemize}
}
