%2-fundamentosTecnicos. Conceptos fundamentales y trabajos relacionados sobre los que se sustenta el proyecto desarrollado. Referenciar trabajos relevantes.

\subsection{Conceptos Fundamentales}
\begin{description}
	\item[JAAS:]
Java Authentication and Authorization Service, es una Interfaz de Programación de Aplicaciones que permite a las
aplicaciones Java acceder a servicios de control de autenticación y acceso.
	\item[JBoss:]
JBoss es un servidor de aplicaciones J2EE de código abierto implementado en
Java puro. Al estar basado en Java, JBoss puede ser utilizado en cualquier
sistema operativo que lo soporte.

JBoss implementa todo el paquete de servicios de J2EE.

%El servidor de aplicaciones JBoss cuenta con utiliza dentro de el las siguientes herramientas:
%\begin{itemize}
%	\item JBoss AOP
%	\item Hibernate
%	\item JBoss Cache
%	\item JBoss IDE
%	\item JBoss jBPM
%	\item JBoss Portal
%	\item JGroups
%	\item Tomcat
%	\item JBoss Mail Server
%	\item JBoss MQ
%	\item JBoss Messaging
%	\item JBoss Forum
%\end{itemize}

	\item[SSO:]
Single sign-on (SSO) es un procedimiento de autenticación que habilita al
usuario para acceder a varios sistemas con una sola instancia de
identificación.

Hay cinco tipos principales de SSO, también se les llama reduced sign on
systems (en inglés, sistemas de autenticación reducida).
\begin{itemize}
	\item Enterprise single sign-on (E-SSO)
	\item Web single sign-on (Web-SSO),
	\item Kerberos
	\item Identidad Federada
	\item OpenID
\end{itemize}
\item[Servlet:] Los servlets son objetos que corren dentro del contexto de un
contenedor de servlets (ej: Tomcat) y extienden su funcionalidad. También
podrían correr dentro de un servidor de aplicaciones, que, además de
contenedor para servlet, tendrá contenedor para objetos más avanzados, como
son los EJB (Tomcat sólo es un contenedor de servlets).

\end{description}

\subsection{Trabajos Relacionados}

A continuación algunos trabajos relacionados con el presente estudio
realizado:

\begin{itemize}
	\item User authentication and authorization in the JavaTM platform.~\cite{paper4}
	\item An operational semantics of Java 2 access control.~\cite{paper1}
	\item A Comparative Study of Access Control Languages.~\cite{paper2}
	\item Automated generation of enforcement mechanisms for semantically-rich security policies in Java-based multi-agent systems.~\cite{paper3}
\end{itemize}
