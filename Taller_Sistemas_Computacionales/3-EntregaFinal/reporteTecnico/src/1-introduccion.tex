%1-introduccion. Explicar el contexto, problema abordado y los objetivos del trabajo. Definir cómo está organizado el informe.

\subsection{Contexto}
El Control de Acceso Lógico a aplicaciones y sistemas es el primer obstáculo a
superar por un “atacante” para el acceso no autorizado a un equipo y a los
datos que contiene. La utilización de métodos de seguridad combinados como la
biometría permite reducir la probabilidad de éxito en los intentos de acceso
por personal no autorizado.

Los Controles de Acceso lógico son de vital importancia ya que permiten
proteger los recursos de los equipos. Esto se acrecienta cuando estos recursos
almacenan información confidencial, documentos clasificados, bases de datos
internas del ejército y/o Ministerio de Defensa de un país.

Este documento es una base para aplicar Control de Acceso Lógico en una máquina
JBoss, para sus aplicaciones pertinentes, específicamente utilizando la API \emph{JAAS}.

\emph{JAAS} son las siglas de \emph{Java Authentication and Authorization}. Se
trata de una especificación integrada en la máquina virtual Java a partir de
la versión 1.4 y cuya finalidad es la de definir un estándar para los procesos
de autentificación y autorización.

Es decir ambos procesos, autentificación y autorización están directamente
relacionados con la seguridad de aplicaciones.

Para contextualizar de una mejor manera definiremos los conceptos de \emph{autentificación} y \emph{autorización}.

La \emph{autentificación} es el proceso por el cual un usuario o servicio tiene que
autentificarse para poder acceder a ciertos servicios que ofrece el sistema.

Existen distintas categorías de autentificación:
\begin{itemize}
	\item \textbf{¿Qué sabes?}: Información que el usuario conoce. Ej: Contraseñas, respuestas a preguntas como: "Nombre de tu mascota".
	\item \textbf{¿Qué tienes?}: Elementos físicos que el usuario posee. Ej: Tarjetas bancarias.
	\item \textbf{¿Quién eres?}: Técnicas biométricas como lectura de retina o huellas dactilares.
\end{itemize}

\newpage

La \emph{autorización} es el proceso por el cual se controlan las acciones que tiene
un usuario o servicio normalmente ya autenticado puede realizar, para ello se le
conceden o deniegan permisos.

Categorías de autorización:
\begin{itemize}
    \item \textbf{Autorización declarativa}: En este tipo de autorización los privilegios son gestionados un administrador independientemente de manera externa al código de la aplicación
    \item \textbf{Autorización programática}: En este tipo de autorización las decisiones de autorización se realizan desde el código fuente de la aplicación.
\end{itemize}

\subsection{Objetivos}

Los objetivos específicos planteados en este informe, para la configuración y funcionamiento de la aplicación JAAS sobre máquina JBoss son los siguientes:

\begin{itemize}
	\item Instalación de API JAAS
	\item Integración de LDAP en JBoss usando JAAS
	\item Integración de JAAS en Tomcat
	\item Implementación de Single Sign-On
\end{itemize}
