%5-resultados. Pruebas, mediciones, resultados y conclusiones

\begin{enumerate}

\item \textbf{Pruebas:} La mayor parte de las pruebas se hicieron cuando se tuvo configurado el
servidor \emph{JBoss} con \emph{JAAS} y \emph{SSO}. La única forma de saber si la configuración fue la correcta, es
con un servlet en el servidor Tomcat que intentase autentificar con usuario y contraseña al
servidor \emph{JBoss}. Esta es la única forma de ver en forma tangible si la configuración tiene todo lo necesario.

\item \textbf{Resultados:} Los resultados obtenidos son varios. En primer lugar fue factible
habilitar autentificación vía \emph{JAAS} en un servidor \emph{JBoss}. Por otro lado se logro que se pudiese
autentificar contra \emph{LDAP} al igual que utilizara el \emph{SSO} para sesiones independientes. Lo único
que no se llevo a cabo fue la integración con los demás grupos, todos los resultados obtenidos
fue en un servidor de pruebas propio.

\item \textbf{Conclusiones:} Si bien no se logro la integración con los demás grupos se usaron
todos los servicios que se requerían, ya sea \emph{LDAP}, \emph{Tomcat} y \emph{JBoss}. Por otro la configuración
de \emph{JAAS} y \emph{SSO} fue bastante simple en el servidor \emph{JBoss}, no así la implementación de un servlet
que integra todo lo visto para poder ver tangiblemente. Este punto fue el mas difícil ya que se
tuvo que programar. 
Finalmente, podemos darnos cuenta que \emph{JAAS} es una herramienta bastante útil en cuanto a
\emph{Control de Acceso Lógico}, para poder restringir el uso de aplicaciones en un servidor determinado;
y si se une con el método \emph{Single Sign-On}, se convierte en una opción totalmente viable para la autenticación
y autorización de usuarios.
\end{enumerate}
