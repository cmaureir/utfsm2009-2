\lstset{basicstyle=\footnotesize, breaklines=true, stringstyle=\ttfamily, numbers=left,
%frame=shadowbox, rulesepcolor=\color{black}}
frame=single, linewidth={16cm}}
	\subsection{Instalación de base de datos LDAP}
	\begin{enumerate}
		\item Bajar e instalar OpenLDAP.
\begin{lstlisting}
$ yum install openldap openldap-devel openldap-servers openldap-clients
\end{lstlisting}
		\item Configurar slapd.conf (hay que especificar el suffix, rootdn (usuario al
cual JBoss se conectará), rootpw, y opcionalmente el directorio.
\begin{lstlisting}
#############################################################
# ldbm database definitions
#############################################################

database	bdb
suffix		"dc=example,dc=com"
rootdn        	"cn=Manager,dc=example,dc=com"
# Cleartext passwords, especially for the rootdn, should
# be avoid.  See slappasswd(8) and slapd.conf(5) for details.
# Use of strong authentication encouraged.
rootpw       	secret
# The database directory MUST exist prior to running slapd AND
# should only be accessible by the slapd and slap tools.
# Mode 700 recommended.
directory	/my/custom/openldap/data/directory
# Indices to maintain
index	objectClass	eq
\end{lstlisting}
			\item Configuramos el puerto a utilizar y especificamos el camino a slapd.conf:
(se recomienda un puerto mayor al 1024 para no tener que usar privilegios de
root)
\begin{lstlisting}
/path/to/slapd -h "ldap://127.0.0.1:10389" -f /path/to/my/custom/slapd.conf
\end{lstlisting}
 Ej:
\begin{lstlisting}
$ /usr/sbin/slapd -h "ldap://127.0.0.1:10389" -f /etc/openldap/slapd.conf
\end{lstlisting}
		\item  Necesitamos tener los necesarios usuarios y grupos en LDAP que serán
utilizados en JBoss. Aca va el ldap-setup.ldif que tiene la estructura que
corresponde a la configuración de ejemplo que utilizaremos con JBoss. Para
esto utilizamos el siguiente comando:
\begin{lstlisting}
$ ldapadd -x -W -D 'cn=Manager,dc=example,dc=com' -f ldap-setup.ldif -c
\end{lstlisting}


\begin{lstlisting}
# Top-level structure of the LDAP hierarchy

dn: dc=example,dc=com
objectclass: top
objectclass: dcObject
objectclass: organization
dc: example
o: Example Corporation

dn: ou=People,dc=example,dc=com
objectclass: top
objectclass: organizationalUnit
ou: People

dn: ou=Groups,dc=example,dc=com
objectclass: top
objectclass: organizationalUnit
ou: Groups

# Users.

# Note that passwords are cleartext here, but an LDAP will allow safer ways
# like using hashes.

# The master user has all privileges: handy during development.

dn: uid=master,ou=People,dc=example,dc=com
objectclass: top
objectclass: uidObject
objectclass: person
uid: master
cn: Master Admin
sn: Admin
userPassword: master

# This admin user has rights to all admin roles: in this case functional admin
# and technical admin.

dn: uid=admin,ou=People,dc=example,dc=com
objectclass: top
objectclass: uidObject
objectclass: person
uid: admin
cn: Admin Nonymous
sn: Admin
userPassword: admin

# A standard user of your system.

dn: uid=member,ou=People,dc=example,dc=com
objectclass: top
objectclass: uidObject
objectclass: person
uid: member
cn: Marcel Ember
sn: Member
userPassword: member

# Groups. 
# Assigning somebody to one or more groups will allow you to do role-based
# authorization in an application.

dn: cn=NormalUser,ou=Groups,dc=example,dc=com
objectclass: top
objectclass: groupOfNames
cn: NormalUser
description: the group of regular users
member: uid=normaluser,ou=People,dc=example,dc=com
member: uid=master,ou=People,dc=example,dc=com

dn: cn=FunctionalAdmin,ou=Groups,dc=example,dc=com
objectclass: top
objectclass: groupOfNames
cn: FunctionalAdmin
description: the role to manage certificates (the RPKI engine)
member: uid=admin,ou=People,dc=example,dc=com
member: uid=master,ou=People,dc=example,dc=com

dn: cn=TechnicalAdmin,ou=Groups,dc=example,dc=com
objectclass: top
objectclass: groupOfNames
cn: TechnicalAdmin
description: the role to manage resources
member: uid=admin,ou=People,dc=example,dc=com
member: uid=master,ou=People,dc=example,dc=com
\end{lstlisting}
	\end{enumerate}

	\subsection{Implementación del SSO en JBoss}
	\begin{enumerate}
		\item Para activar SSO, simplemente hay que descomentar la siguiente
linea del archivo de configuración (TomCat)
JBOSS\_HOME/server/tu\_configuracion/deploy/jboss-web.deployer/server.xml:
\begin{lstlisting}
<Valve className="org.apache.catalina.authenticator.SingleSignOn" /> 
\end{lstlisting}

	\end{enumerate}
	\subsection{Integración de LDAP utilizando JAAS en servidores web JBoss}
	\begin{enumerate}
			\item JBoss usa el LoginLdapModule para conectarse con LDAP.
			Configuramos la application-policy en el login-config.xml que puede ser encontrado en JBOSS\_HOME/server/default.conf. Esto permitirá a JBoss conectarse con el servidor LDAP y le dirá la estructura que debe tener, como su usuario, password y los roles que encontrará.
\begin{lstlisting}
<?xml version='1.0'?>
<!DOCTYPE policy PUBLIC
      "-//JBoss//DTD JBOSS Security Config 3.0//EN"
      "http://www.jboss.org/j2ee/dtd/security_config.dtd">

<policy>
	<application-policy name = "mysecuritydomain">
	    <authentication>
	        <login-module code="org.jboss.security.auth.spi.LdapLoginModule" flag="required">
	            <module-option name="java.naming.factory.initial">com.sun.jndi.ldap.LdapCtxFactory</module-option>
	
	            <module-option name="java.naming.provider.url">ldap://127.0.0.1:10389</module-option>
	            <module-option name="java.naming.security.authentication">simple</module-option>
	
				<module-option name="java.naming.security.principal">cn=Manager,dc=ripe,dc=net</module-option>
				<module-option name="java.naming.security.credentials">secret</module-option>
				
	            <module-option name="principalDNPrefix">uid=</module-option>
	            <module-option name="principalDNSuffix">,ou=People,dc=example,dc=com</module-option>
	            <module-option name="rolesCtxDN">ou=Groups,dc=ripe,dc=net</module-option>
	            <module-option name="uidAttributeID">member</module-option>
	            <module-option name="matchOnUserDN">true</module-option>
	            <module-option name="roleAttributeID">cn</module-option>
	            <module-option name="roleAttributeIsDN">false</module-option>
	            <module-option name="searchTimeLimit">5000</module-option>
	
	            <module-option name="searchScope">ONELEVEL_SCOPE</module-option>
	        </login-module>
	    </authentication>
	</application-policy>
</policy>
\end{lstlisting}
			\item Después de esto solo nos queda conectarnos a través del código con JAAS:\\
			\textit{Este código chequea el usuario/password y obtiene los roles del usuario a traves de JAAS.}
\begin{lstlisting}
package net.ripe.certification.ui;

import java.io.IOException;
import java.security.Principal;
import java.security.acl.Group;
import java.util.Enumeration;

import javax.security.auth.Subject;
import javax.security.auth.callback.Callback;
import javax.security.auth.callback.CallbackHandler;
import javax.security.auth.callback.NameCallback;
import javax.security.auth.callback.PasswordCallback;
import javax.security.auth.callback.UnsupportedCallbackException;
import javax.security.auth.login.LoginContext;
import javax.security.auth.login.LoginException;
import javax.servlet.http.HttpServletRequest;

import org.apache.log4j.Logger;
import org.apache.wicket.Request;
import org.apache.wicket.authentication.AuthenticatedWebApplication;
import org.apache.wicket.authentication.AuthenticatedWebSession;
import org.apache.wicket.authorization.strategies.role.Roles;

public class JAASBasedSession extends AuthenticatedWebSession {
	
	public static final String ROLES_GROUP_NAME = "Roles";

	public static final String JBOSS_APPLICATION_POLICY_NAME = "mysecuritydomain";

	private final static Logger log = Logger.getLogger(JAASBasedSession.class);
	
	private Subject subject;
	private Roles roles = new Roles();
	
    @SuppressWarnings("deprecation")
	public JAASBasedSession(AuthenticatedWebApplication app, Request request) {
        super(app, request);
    }

    public boolean authenticate(String username, String password) {
    	if (log.isDebugEnabled()) {
			// SECURITY WARNING: The password is logged in the clear here. Handy to get all this to work, but remove ASAP.
    		log.debug("Entering JAASBasedSession.authenticate(" + username + ", " + password + ")");
    	}

    	boolean authenticated = false;
    	LoginCallbackHandler handler = new LoginCallbackHandler(username, password);
    	try {
    		LoginContext ctx = new LoginContext(JBOSS_APPLICATION_POLICY_NAME, handler);
			ctx.login();
			authenticated = true;
			subject = ctx.getSubject();
			
			for (Principal p : subject.getPrincipals()) {
				if (log.isDebugEnabled()) {
					// Watch this debug line if something doesn't work for you: the structure that is returned might
					// differ in naming and structure. The JBoss LdapLoginModule returns two principals. One has the
					// user itself, the other is called "Roles", is a Group and holds the role names that the user
					// is authorized for (i.e. belongs to in the LDAP).
					log.debug("Principal for " + username + ": " + p.toString());
				}
				
				// Group is a subclass of Principal, and the members are the names of the roles
				// as set up in the LDAP. In my example this is the cn attribute of an LDAP Group.
				if ((p instanceof Group) && (ROLES_GROUP_NAME.equalsIgnoreCase(p.getName()))) {
					Group g = (Group) p;
					Enumeration<? extends Principal> members = g.members();
					while (members.hasMoreElements()) {
						Principal member = members.nextElement();
						roles.add(member.getName());
						if (log.isDebugEnabled()) {
							log.debug("Added role " + member.getName() + " for username " + username + ".");
						}
					}
				}
			}
		} catch (LoginException e) {
			// You'll get a LoginException on a failed username/password combo.
			authenticated = false;
		}
        return authenticated;
    }

    public Roles getRoles() {
        return roles;
    }

    private class LoginCallbackHandler implements CallbackHandler {

    	private String username;
    	private String password;
    	
    	public LoginCallbackHandler(String username, String password) {
    		this.username = username;
    		this.password = password;
    	}
    	
		public void handle(Callback[] callbacks) throws IOException, UnsupportedCallbackException {
	    	for (int i = 0; i < callbacks.length; i++) {
				Callback callback = callbacks[i];
				if (callback instanceof NameCallback) {
					((NameCallback) callback).setName(username);
				} else if (callback instanceof PasswordCallback) {
					PasswordCallback pwCallback = (PasswordCallback) callback;
					pwCallback.setPassword(password.toCharArray());
				} else {
					throw new UnsupportedCallbackException(callbacks[i], "Callback type not supported");
				}
			}
		}
    	
    }
}
\end{lstlisting}
		\item Servlet de prueba:
\begin{lstlisting}
import java.io.IOException;
import javax.servlet.*
import javax.servlet.http.*;
import javax.security.auth.login.*;
import javax.security.auth.callback.*;
import java.util.logging.Logger;
 public class JAASLoginFilter implements Filter
 {
  private final static Logger log = Logger.getLogger(JAASLoginFilter.class.getName());
   
  public void doFilter(ServletRequest request, ServletResponse response, FilterChain chain) throws IOException, ServletException
   {
    login(request);
    chain.doFilter(request, response);
    logout(request);
   }
  public void init(FilterConfig filterConfig) {}
  public void destroy() {}
 }
 
 static class LoginCallback implements CallbackHandler
 {
  private String username;
  private String password;
  
  protected LoginCallback(String username, String password)
  {
   this.username = username;
   this.password = password;
  }

  public void handle(Callback[] callbacks)
  {
   for(int i=0; i < callbacks.length; i++) 
   {
    if(callbacks[i] instanceof NameCallback)
    {
     NameCallback nameCallback = (NameCallback)callbacks[i];
     nameCallback.setName(username);
    } 
    else if(callbacks[i] instanceof PasswordCallback) 
    {
     PasswordCallback passwordCallback = (PasswordCallback)callbacks[i];
     passwordCallback.setPassword(password.toCharArray());
    }
   }
  }
 }

 public static void login(ServletRequest request)
 {
  // Don't do anything if there is no session
  
  if(!(request instanceof HttpServletRequest)) 
   return;
  HttpSession session = ((HttpServletRequest)request).getSession(false);
  if(session == null)
   return;
 
  // Get username and password from the session

  String username = (String)session.getAttribute("username");
  String password = (String)session.getAttribute("password");

  // Don't login if the username or password isn't set

  if(username == null || password == null)
   return;
  
  // Login
CallbackHandler handler = new LoginCallback(username, password);
LoginContext lc = null;
try {
lc = new LoginContext("clientlogin",
handler);
lc.login();
request.setAttribute("jaaslogincontext",
lc);
request.setAttribute("jaasusername",
username);
} catch (LoginException e) {
log.warning("error logging in, username=" + username + " : " + e.getMessage());
}
}
public static void logout(ServletRequest request)
{
LoginContext lc = (LoginContext)
request.getAttribute("jaaslogincontext");
String username = (String)
request.getAttribute("jaasusername");
if(lc == null || username == null) return;
// Logout
try {
lc.logout();
} catch (LoginException e) {
log.warning("error logging out, username=" + username + " : " + e.getMessage());
}
}
\end{lstlisting}
	\end{enumerate}
