%B. REPROGRAMACIÓN DE ACTIVIDADES

%1. De acuerdo a lo realizado y los problemas identificados anteriormente, revise
%sus objetivos y redefínalos con un criterio realista de cumplimiento. Si estima
%necesario, reenvíe una nueva versión de la propuesta con todas las
%correcciones hechas.
\begin{enumerate}
	\item En cuanto a los objetivos y tareas a realizar creemos que estamos en buen camino,
		teniendo en cuenta que los tiempos que quedan para poder terminar el proyecto son los correctos
		y tenemos la convicción de terminar lo antes posible.

		El problema que se discutió anteriormente se debe a que no conocemos la forma de implementación del servidor,
		pero creemos que poseen una configuración estándar, por lo que nuestra implementación no debería tener problema alguno.
		
		Un punto a favor es la experiencia en servidores de distinto tipos, que cada uno de los integrantes de nuestro equipo posee,
		lo cual creemos que es fundamental para poder solucionar los futuros problemas de implementación.

%2. Defina las actividades a desarrollar para terminar su proyecto.

	\item Según lo planeado por el equipo, las actividades que quedan por desarrollar son:
	\begin{itemize}
		\item \emph{Integración de soluciones con Servidores Web y Servidor JBoss}:

			Este es el último paso de nuestro proyecto, siendo el hito que le da sentido al proyecto realizado.
			Debemos fijar una instancia para poder integrar nuestras configuraciones con los grupos señalados..
			Con esto nuestro proyecto debiera concluir otorgando mayor seguridad y cuentas individuales a los usuarios de los servicios.

		\item \emph{Documentación final}:

			Debemos redactar la documentación final de nuestro proyecto, para que sea una guía útil en caso de que sea necesario
			implementar un Control de Acceso Lógico en servidores JBoss.

			Se verá la posibilidad de escribir un Reporte Técnico para el Departamento de Informática de nuestra universidad,
			con el fin ser un aporte en nuestra carrera.			
	\end{itemize}
\end{enumerate}
