%REVISIÓN DEL PLAN DE TRABAJO

%1. Refiérase brevemente a las actividades que se planificaron para el período a
%cumplirse en el próximo hito (informe de avance, 5 de Octubre) e indique qué
%grado de cumplimiento tiene respecto a lo planificado inicialmente. Resalte los
%principales avances logrados.

\begin{enumerate}
	\item Para el presente hito, el grupo de \emph{Control de Acceso Lógico},
		realizó las actividades que quedaron pendientes según el informe de avance
		del 4 de Octubre.\\
		Las tareas que se realizaron son:
		\begin{enumerate}
			\item Implementación de SSO (Single Sing On) en JBOSS. 
			\item Integración de JAAS con Servidor Web (Tomcat).
		\end{enumerate}  

		Durante el proceso de implementación de éstas soluciones,
		el equipo se centró en poner en práctica lo investigado,
		siendo crucial para poder integrar SSO a JBOSS y JAAS con Tomcat.
		Por otro lado tuvimos que implementar nuestro propio Servidor Tomcat para hacer las pruebas pertinentes,
		antes de poder implementarlo en el servidor del equipo de \emph{Servidores Web}.

		En primer lugar, hicimos la integración SSO, que resulto ser una tarea simple
		debido a que la integración consta sólo de una activación en el servidor.
		En nuestro caso tuvimos que editar la configuración del servidor, del perfil que estábamos
		utilizando.
		\begin{verbatim}
		[root@tsc ~]# vim jboss-5.1.0.GA/server/default/deploy/jbossweb.sar/server.xml
		\end{verbatim}
		Luego des-comentamos la siguiente línea:
		\begin{verbatim}
		<Valve className="org.apache.catalina.authenticator.SingleSignOn" />
		\end{verbatim}
		Finalmente tenemos que reiniciar nuestro servidor JBoss,
		para que la configuración tenga efecto.

		Para comprobar que realmente SSO esta funcionando,
		basta con realizar una prueba con una pagina web en jsp y un login,
		información que tenemos documentada en nuestro TRAC.

		Respecto al segundo punto, aunque es posible utilizar JAAS en Tomcat como
		un mecanismo de autenticación (JAASRealm), la flexibilidad del marco de JAAS se pierde una vez que el usuario se autentica.
		Esto es porque los principios usados para designar al ``usuario'' y
		``rol'', ya no están disponibles en el contexto de seguridad
		en el que se ejecuta la webapp.
		Esto reduce el marco de JAAS para fines de autorización a un simple usuario/sistema de roles, que pierde su conexión con la
		política de seguridad de Java.
		
		En nuestro intento por integrar la autenticación JAAS con Tomcat ponemos una aplicación plenamente utilizando autorización
		JAAS en su lugar, utilizando un par de trucos para hacer frente a algunas de las idiosincrasias de Tomcat. 

	\item Hasta el momento, se han realizado las partes más importantes del proyecto,
		como son la implementación de SSO en JBOSS y la integración con un servidor Web, Tomcat.
		El único punto que no alcanzamos a realizar fue la integración con el grupo de \emph{Servidores Web},
		ya que probamos la solución para poder demostrar que esta funcionando SSO y así asegurarnos que la integración
		funciona correctamente antes de implementarla en los servidores del equipo asignado.

	\item Durante la realización de la presente etapa, no encontramos actividades extras para poder cumplir con las tareas
		propuestas anteriormente. La única dificultad que tuvimos fue tener que levantar nuestro propio servidor Tomcat
		y programar un \emph{servlet} para poder comprobar que nuestras configuraciones estuvieran trabajando realmente.

	\item Para la última parte del proyecto nos queda integrar todo lo hecho en nuestros servidores, con los servicios
		desarrollados por los grupos \emph{Servidor JBOSS} y \emph{Servidores Web}. Esto será una tarea, mas o menos, rápida,
		pero no libre de problemas, ya que si las configuraciones son distintas a las de nuestros servidores tendremos que cambiar
		las presentes implementaciones para que puedan desenvolverse en los servidores correspondientes.
		Como medida correctiva, antes de implementar nuestras soluciones en los servidores de los grupos antes mencionados,
		buscaremos que nuestras soluciones sean lo mas alto nivel posible para que puedan ser implementadas sin ocurrir ningún error,
		razón por la cual poseemos una documentación completa, tanto con documentos realizados por nosotros como externos.

\end{enumerate}
