%En un máximo de 2 páginas formule la planificación de su proyecto, donde usted se debe referir a
%los siguientes puntos.

\subsection{Metodología de Trabajo}
	%Señale los enfoques y procedimientos a ser usados en el desarrollo del proyecto. Especifique cómo
	%piensa trabajar para lograr todos los objetivos propuestos. Refiérase a cómo se organizará el
	%equipo de trabajo, responsabilidades asumidas por cada miembro, y qué medios de comunicación
	%y de toma de decisiones emplearán.


Nuestro equipo ha trabajado anteriormente en muchas otras actividades, tanto académicas como con otros
proyectos, por lo que para el desarrollo del proyecto pensamos continuar usando la siguiente metodología
de trabajo:
\begin{description}
	\item[\textbf{Horarios Semanales de trabajo:}] 
		Se coordinarán horarios de trabajo en grupo para asegurar tiempo de trabajo en el proyecto
		durante la semana.

	\item[\textbf{Reuniones presenciales:}]
		Cada semana nos reuniremos en un horario definido a revisar los avances de cada uno y resolver
		los problemas que vayan ocurriendo.

	\item[\textbf{Coordinación y asignación de metas semanales:}]
		Cada semana se asignarán las metas y tareas a cumplir cada semana, las cuales serán evaluadas
		durante las reuniones presenciales.

	\item[\textbf{Worklogs:}]
		Se deberá dejar un registro de toda actividad realizada, indicando el día y
		el detalle de lo realizado en el Trac habilitado para el proyecto.

\end{description}



\subsection{Plan de Trabajo}
	%Coherentemente con la definición de sus objetivos y la metodología de trabajo, identifique las
	%tareas que desarrollará en el contexto del proyecto. Refiérase a la distribución y asignación de
	%tareas entre los miembros del equipo. Construya una carta Gantt con las principales tareas y los
	%hitos que debe cumplir en el desarrollo del proyecto.
En el presente plan de trabajo, se enlistan las actividades a realizar en este proyecto y sus participantes:
\small
\begin{itemize}
\item Investigación \emph{(Todos)}
\begin{itemize}
	\item JBOSS
	\item Documentación Básica
	\item Uso y aplicación de JAAS en JBOSS
	\item JAAS 
	\item Documentación Básica
	\item Distintos tipos de Uso
	\item Funcionamiento Cliente-Servidor
	\item Sign-On
	\item Políticas de Seguridad
	\item LDAP
	\item Integración LDAP con JAAS
\end{itemize}
\item Implementación
\begin{itemize}
	\item Implementación simple de plataforma JBOSS \emph{(Rodrigo Fernández y Cristian Maureira)}
	\item Implementación simple de protocolo LDAP \emph{(Gabriel Zamora y Cristian Maureira)}
	\item Implementación de Single Sign-On en JAAS \emph{(Rodrigo Fernández y Ignacio Villacura)}
	\item Implementación de servicio JAAS \emph{(Ignacio Villacura y Esteban Bombal)}
	\item Desarrollo y uso de JAAS en JBOSS usando LDAP \emph{(Gabriel Zamora y Esteban Bombal)} 
	\item Desarrolo servicio web sobre JBOSS \emph{(Rodrigo Fernandez e Ignacio Villacura)}
	\item Integración de JAAS y LDAP \emph{(Cristian Maureira y Esteban Bombal)}
	\item Integración de servicio web y JAAS \emph{(Gabriel Zamora y Esteban Bombal)}
	\item Testing y Debugging \emph{(Rodrigo Fernandez)}
\end{itemize} 
\item Integración entre proyectos 
\begin{itemize}
	\item Integración con proyecto LDAP \emph{(Cristian Maureira y Esteban Bombal)}
	\item Integración con proyecto JBOSS \emph{(Gabriel Zamora e Ignacio Villacura)}
\end{itemize}
\item Entregables \emph{(Rodrigo Fernández)}
\begin{itemize}
	\item Primer Informe de Avance (5-10-09) 
	\item Segundo Informe de Avance (2-11-09) 
	\item Informe de Gestión del Proyecto 
	\item Reporte Técnico
\end{itemize} 
\item Otros 
\begin{itemize}
	\item Mantención Bitácora 
	\item Mantención Carta Gantt 
\end{itemize}
\end{itemize}


Para mas detalle, ver la Carta Gantt en el anexo.\\


\subsection{Recursos}
	%Para el desarrollo de cualquier proyecto es necesario disponer de recursos tales como
	%documentación bibliográfica, software, hardware, y otros. En el caso particular de su proyecto,
	%especifique los recursos necesarios que ustedes requieren para desarrollarlo. Identifique los
	%recursos que ya tienen disponibles y cuáles deben adquirir.

A continuación, se identifican los principales recursos técnicos a utilizarse durante el desarrollo.

Para la coordinación y comunicación del equipo, se utilizarán las siguientes herramientas:

\begin{itemize}
    \item Alias de Correo, Trac y Skype
\end{itemize}

El proyecto se desarrollará completamente sobre la plataforma Linux,
específicamente las plataformas Fedora y CentOS que son utilizada en los laboratorios en el cual el
equipo trabaja, el laboratorio de Sistemas Distribuidos y el laboratorio de Integración Tecnológica.

Los lenguajes de programación y asociados que serán utilizados son los siguientes:

\begin{itemize}
    \item  Java Platform, Enterprise Edition (JEE).
\end{itemize}

Las herramientas que se usarán para el desarrollo son en su totalidad aplicaciones OpenSource.
Se destacan las siguientes:

\begin{itemize}
    \item LDAP, Git, LaTeX, Vi IMproved (VIM) y Planner
\end{itemize}

Por último, el hardware a utilizar será 1 servidor del LabIT o LabSD para poder llevar a cabo las pruebas
del proyecto.

