%%% Esqueleto base de la presentacion
%%% No agregar las paginas con un include
%%% Lo que quieran aportar deben ser usuarios del TRAC

\documentclass{beamer}
\usepackage[spanish,activeacute]{babel}
\usepackage[utf8]{inputenc}
\usepackage{listings}
\usepackage{color}
\usepackage{graphics}
\definecolor{gray2}{rgb}{100,100,100}

\usetheme[pageofpages=of,% String used between the current page and the
                         % total page count.
          alternativetitlepage=true,% Use the fancy title page.
          titlepagelogo=img/logo,% Logo for the first page.
          watermark=img/di-off,% Watermark used in every page.
          watermarkheight=100px,% Height of the watermark.
          watermarkheightmult=4,% The watermark image is 4 times bigger
                                % than watermarkheight.
          ]{Torino}

\usecolortheme{nouvelle}

\author{\normalsize
\textbf{Integrantes:}\\
Cristian Maureira\\
Rodrigo Fernández\\
Ignacio Villacura\\
Gabriel Zamora\\
\vspace{0.2cm}
\textbf{Jefe de Proyecto:}\\
Esteban Bombal\\
\textcolor{gray}{ebombal@alumnos.inf.utfsm.cl}
}
\title{\Huge Control de Acceso Lógico}
\subtitle{\Large \textit{``Informe de Avance I''}}
\institute{Universidad Técnica Federico Santa María}
%\date{\today}

\begin{document}
\begin{frame}[t,plain]
\titlepage
\end{frame}


\section{Resumen}

\frame{
\frametitle{Resumen}

Implementación de un framework de seguridad basado en Java, llamado \textbf{JAAS}. 
Establecer una conexión a servicios de directorios mediante \textbf{LDAP} para la conexión a la máquina \textbf{JBoss} correspondiente.

}

\section{Revisión del plano de trabajo}
\frame{
\frametitle{}
\begin{center}
	\Huge Revisión del plano del trabajo
\end{center}
}

\frame{
\frametitle{Organización}
\begin{itemize}
	\item Actividades Carta Gantt (8 sept. - 20 nov.)
	\item Áreas centrales de las actividades:
	\begin{itemize}
		\item Investigación
		\item Implementación
	\end{itemize}
\end{itemize}
}

\frame{
\frametitle{}
\begin{center}
	\Huge Proceso de Investigación
\end{center}
}

\frame{
\frametitle{Proceso de Investigación}
\begin{itemize}
	\item Comprender funcionamiento de la autentificación mediante \emph{Java Authentication and Authorization Service} (JAAS)
	\item Analizar la posibilidad de integrar JAAS con Single Sign-On (SSO)
	\item Confirmar la viabilidad de una futura integración con otros proyectos del ramo
	\item Capacitación con respecto al funcionamiento e implementación de servidores \emph{JBoss}
	\item Estudio de la integración entre JAAS y JBoss
\end{itemize}
}

\frame{
\frametitle{}
\begin{center}
	\Huge Proceso de Implementación
\end{center}
}

\frame{
\frametitle{Proceso de Implementación}
\begin{itemize}
	\item Máquina de trabajo: (LabComp)
	\begin{itemize}
		\item Fedora 11 - Leonidas
		\item 512M RAM
		\item 2.80GHz CPU
	\end{itemize}
	\item Implementación de un servidor \emph{JBoss}
	\item Implementación de un servicio de directorio \emph{LDAP}
	\item Configuración de \emph{JAAS}
	\item Integración de \emph{LDAP} con \emph{JAAS} en el servidor \emph{JBoss}
\end{itemize}
}

\frame{
\frametitle{Aspectos sobre la planificación}
\begin{itemize}
	\item Cumplimiento de la Carta Gantt
	\item Punto pendiente: Probar la implementación de SSO con JAAS
	\item Ausencia de actividades no planificadas
	\item Futura Integración con el grupo de \emph{Servicio Web}
\end{itemize}
}


\section{Reprogramación de actividades}
\frame{
\frametitle{Reprogramación de actividades}
\begin{itemize}
	\item Correcta estimación de tiempos para el trabajo futuro
	\item Actividades restantes:
	\begin{itemize}
		\item Implementación de SSO con JAAS
		\item Integración de JAAS con Servidores Web
		\item Integración de proyectos del ramo
	\end{itemize}
\end{itemize}
}

\frame{
\frametitle{}
\begin{center}
	\Huge ¿Preguntas?
\end{center}
}
\end{document}
