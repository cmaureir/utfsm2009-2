%B. REPROGRAMACIÓN DE ACTIVIDADES

%1. De acuerdo a lo realizado y los problemas identificados anteriormente, revise
%sus objetivos y redefínalos con un criterio realista de cumplimiento. Si estima
%necesario, reenvíe una nueva versión de la propuesta con todas las
%correcciones hechas.
\begin{enumerate}
\item En cuanto a objetivos, consideramos que las metas propuestas y tiempos
estimados son los correctos para lograr con éxito el proyecto planteado
para este semestre.

Los problemas identificados anteriormente se debieron
más que nada a una subestimación de los trabajos a realizar, dejando para
etapas finales la ayuda de los otros grupos involucrados.\\

%2. Defina las actividades a desarrollar para terminar su proyecto.

\item Según lo que el equipo planeó en la etapa preliminar al proyecto, las
actividades restantes son:
\begin{itemize}
	\item \emph{\textbf{Implementación de Single Sign-On en JAAS}}:
		\emph{Single Sign-On} (SSO) es un procedimiento de autenticación que
		permite al usuario acceder a distintos sistemas a través de un único
		proceso de identificación. Se pretende integrar este procedimiento al
		proyecto de manera de simplificar el sistema de identificación para los
		usuarios de máquinas \emph{JBoss}.
	\item \emph{\textbf{Integración de JAAS con Servidores Web}}:
		Dentro de los objetivos se encuentra implementar la autenticación 
		\emph{JAAS} en Servidores Web. Para esto se usará específicamente 
		el servidor web \emph{Tomcat}, a cargo del equipo de Servidores Web.
		Para esto nos comunicaremos con ellos, viendo la posibilidad de que
		se nos proporcione un servidor Web de estas características.
	\item \emph{\textbf{Integración de proyectos Taller de Sistemas
		de Computación}}: Finalmente, para darle sentido al proyecto realizado,
		se necesita de una integración \emph{real} entre los proyectos
		involucrados en el ramo de \emph{Taller de Sistemas de Computación}, 
		por lo que se buscará una instancia con los demás equipos para esta
		labor.
\end{itemize}
\end{enumerate}
