%REVISIÓN DEL PLAN DE TRABAJO

%1. Refiérase brevemente a las actividades que se planificaron para el período a
%cumplirse en el próximo hito (informe de avance, 5 de Octubre) e indique qué
%grado de cumplimiento tiene respecto a lo planificado inicialmente. Resalte los
%principales avances logrados.

\begin{enumerate}
\item Para el presente hito, el grupo de \emph{Acceso Lógico}, planificó una Carta Gantt con
actividades desde el día Martes 8 de Septiembre al Viernes 20 de Noviembre.

Según la planificación nos referiremos a las actividades realizadas hasta el día Domingo
4 de Noviembre.

Destacamos dos partes dentro de las actividades realizadas durante este periodo:
\begin{itemize}
	\item Investigación 
	\item Implementación
\end{itemize}

Durante el proceso de \textbf{Investigación} el grupo se centro en comprender
el funcionamiento de autentificación  mediante \emph{JAAS}, los beneficios que
dicha aplicación nos ofrece y la escalabilidad que otorga, entre ellos el
\emph{Single Sign-On}.

Además fue necesario realizar una capacitación en el tema que consta
la realización de una integración con los demás grupos, es decir, fue necesario
poder comprender el funcionamiento de las distintas aplicaciones que nuestros
compañeros se encuentran realizando, como lo fue el caso de \emph{JBoss}.

Respecto al último punto señalado anteriormente, fue necesario que como
equipo de trabajo pudiéramos configurar un servidor \emph{JBoss},
por lo que, el tiempo de capacitación fue un aspecto fundamenta para
poder lograrlo exitosamente.
Otro punto que no podíamos dejar de lado, era el investigar sobre
la posibilidad de configurar \emph{JAAS} en nuestra futura
máquina con \emph{JBoss}

Finalmente se investigó sobre \emph{LDAP} y como relacionarlo con la autenticación
\emph{JAAS}.

Luego, siguiendo con el proceso de \textbf{Implementación}, etapa que aún está en
desarrollo, comenzamos a trabajar de lleno en la posibilidad de poner en práctica
lo investigado y documentado previamente el medio designado para ello, el \emph{Trac}
de nuestro equipo.

Para ello, primero se dispuso de un servidor ubicado en el Laboratorio de
Computación para realizar todas las pruebas pertinentes al trabajo. Se tuvo que
montar una maquina \emph{JBoss} por nuestra cuenta para comenzar a trabajar
inmediatamente en la Autenticación.

Fue también necesario una base de datos \emph{LDAP}, por lo que tuvimos que instalar
e ingresar datos de prueba para poder realizar la integración de \emph{LDAP} con
\emph{JAAS} en un servidor \emph{JBoss}.\\
 
%2. Dé razones de porqué no se realizaron algunas actividades planificadas (o parte
%de ellas), en caso de existir esta situación.

\item Hasta el momento, y siguiendo la Carta Gantt, el único punto que no se pudo
realizar, fue probar la implementación de \emph{Single Sign-On} en la
Autenticación \emph{JAAS}. No contamos con las dificultades de instruirnos a fondo
e integrar los elementos base para poder implementar \emph{JAAS} (\emph{JBoss} y
\emph{LDAP}), por los que nos demoró poder tener la base para poder trabajar
directamente en el servidor de aplicaciones de \emph{Java}.\\

%3. Identifique las actividades que se realizaron y que no estaban inicialmente
%planificadas. Justifique brevemente su inclusión.

\item Durante la realización de este primer avance, no se encontraron actividades no
planificadas. Todo lo hecho estuvo previsto al momento de realizar la Carta Gantt,
por lo que no nos encontramos con sorpresas que agregaran trabajo al proyecto. 
Aunque, expresado en el punto anterior, no estuvo planificado la dificultad y el
tiempo que nos iba a demorar la implementación de la plataforma \emph{JBoss} y de
la base de datos \emph{LDAP}.\\

%4. Identifique potenciales problemas para el próximo período de trabajo y
%proponga medidas correctivas.

\item Entre algunos problemas que podrían surgir en el próximo periodo de trabajo
está, la integración con el servicio Web, pues si queremos lograr una exitosa
integración de nuestro proyecto con el grupo de Servicios Web, necesitamos
previamente hacer pruebas en nuestro servidor de trabajo. Por esto, es que no
podemos estimar que tan complicado será dicha integración.

Como medida correctiva, nos comunicaremos con el mencionado grupo para recibir
asistencia al momento de implementar este servicio en nuestro servidor de prueba,
para no encontrarnos con los problemas que tuvimos durante el primer avance en
materia de \emph{JBoss} y \emph{LDAP}.
\end{enumerate}
